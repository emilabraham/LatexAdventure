\documentclass{article}
\setlength{\parskip}{6pt}
% I must use this package because Latex does not indent the first paragraph
% after a section by default. In this document, an indentation looks much nicer
% after a section because it makes the document look much more consistent.
\usepackage{indentfirst}
\begin{document}

%I ignore the paragraph skip in this section because it is a heading. Note that
%I also include the noindent tag because latex indents paragraphs by default.
{\parskip=0pt
\noindent Emil Abraham\par
\noindent BIOL 1113\par
\noindent NUID: 001187154}

\section*{Exam 2 Redemption}

\textbf{Question 3.} \textit{The correct answer is a.} Answer a is correct
because the archenteron is the primitive gut that follows the hole, known as the
blastopore. During early development of bilateria, the blastopore develops into
the mouth for protostomes and into the anus for deuterostomes.

\textbf{Question 4.} \textit{The correct answer is d.} Answer d is correct
because a Statocyst is a balance sensory receptor present in some aquatic
invertebrates, like bivalves. This is not used for feeding. All of the other are
for feeding. A stylet is a primitive piercing mouthpart found in some nematodes.

\textbf{Question 6.} \textit{The correct answer is c.} Answer c is correct
because torsion is the rotation of the visceral mass in \textit{larval}
gastropods. Not adults.

\textbf{Question 7.} \textit{The correct answer is c.} Answer c is correct
because a plant's phytomere is a modular unit of a plant. In essence, I like to
think of it as a branch of a tree. It consists of a leaf, an internode which
connects different stem nodes together, and an axillary bud at the base of a
leaf.

\textbf{Question 14.} \textit{The correct answer is c.} Answer a is incorrect
because Planaria and comb jellies are acoelomates. Earthworms and Leeches are
segmented. That leave nematodes as the non-segmented pseudocoelomates.

\textbf{Question 17.} \textit{The correct answer is e.} Answer d is incorrect
because protostomes undergo spiral cleavage during embryogenesis. Protostomes
undergo determinate cleavage, which means that cell's fate is decided early on.
They become specialized and cannot change.

\textbf{Question 18.} \textit{The correct answer is a.} Answer d is incorrect
because flame cells are specialized excratory cells in platyhelminthes. Shiny
spheres are specialized cells in Placozoans that play an anti-predatory role.

\textbf{Question 22.} \textit{The correct answer is c.} Answer a is incorrect
because Veliger is the larval stage of Bivalves. Miracidia is the larval stage
of the flatworms.

\textbf{Question 23.} \textit{The correct answer is b.} Answer c is incorrect
because Miracidia is the larval stage of Trematoda. Planulae is the larval stage
of Cnidarians.

\textbf{Question 24.} \textit{The correct answer is a.} Answer b is incorrect
because Plaulae is the larval stage of Cnidarians. Veliger is the larval stage
of Bivalves.

\textbf{Question 27.} \textit{The correct answer is e.} Answer a is incorrect
because microsporophylls are in fact modified leaves in seedless vascular plants
that bear the microsporangia. The pollen tube cells does divide into 2 sperm
cells. The male gametophytes produce pollen as their gametes. And antheridia
produce motile sperm. So none of the statements are false.

\textbf{Question 28.} \textit{The correct answer is d.} Answer c is incorrect
because Ctenophores do not have bilateral symmetry. Coral are part of bilateria
and have bilateral symmetry.

\textbf{Question 42.} \textit{The correct answer is d.} Answer e is incorrect
because Cercariae are a larval stage in a fluke that travels from intermediate
host to intermediate host. Chaetae is the correct answer because they are called
seta in earthworms which perform oligochaete movement. Chaetae are bristles that
allow these organisms to drag themselves along the ground.

\textbf{Question 50.} \textit{The correct answer is b.} Answer a in incorrect
because the auricles are also used to sense touch. Epitokes use eyespots to
detect when they reach the surface to release eggs and sperm into the water.

\textbf{Question 53.} \textit{The correct answer is b.} Answer d is incorrect
because a male octopus must insert his hectocotylus into the mantle cavity of
the female. Broadcast spawns involves shooting the eggs and sperm into the
water. So they don't stay in the female. Cephalopods flash colors in order to
determine if a partner is ready to mate.

\textbf{Question 54.} \textit{The correct answer is d.} Answer a ins incorrect
because Strobili are found in land plants. They are used to hold
sporangia-bearing stems. Protonemata can be thalloid, which are broad and
leaf-like.

\textbf{Question 56.} \textit{The correct answer is b.} Answer c is incorrect
because intertidal is the sea shore that is covered during high tide and
uncovered during low tide. Pelagic refers to the area of ocean not near the
bottom or near the shore. So that means open ocean.

\textbf{Question 57.} \textit{The correct answer is d.} Answer e is incorrect
because most seed plants are homosporous, meaning they produce only one type of
spore.

\textbf{Question 59.} \textit{The correct answer is c.} Answer b is incorrect
because blastulation is when the blastocoele is formed. The coelom is used with
the hydrostatic skeleton in annelids for transportation.

\textbf{Question 64.} \textit{The correct answer is c.} Answer a is incorrect
because bivalves don't even have a longitudinal muscle. The correct answer is
Adductor muscle because they use a combination of blood pressure and the
adductor muscle to borrow into the sand.

\textbf{Question 65.} \textit{The correct answer is c.} Answer b is incorrect
because the charophytes don't have exine as a root-like structure. That is not a
way they are similar to land plants. Exine is used as a protective shell to
protect pollen grains.

\end{document}
