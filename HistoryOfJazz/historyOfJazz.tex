\documentclass[12pt]{article}
\usepackage[a4paper,left=1in,right=1in,top=1in,bottom=1in]{geometry}
\usepackage{setspace} \begin{document} \noindent Emil Abraham\\ NUID:001187154\\

\centerline{\emph{A Brief History of Jazz}}

\doublespacing The earliest can be traced back to 1910 or very early 1920's in
the city known as "The Birthplace of Jazz", New Orleans. Although the first
official recording of instrumental jazz occurred in New York City by the
Original Dixieland Jazz Band , New Orleans really embraced and provided a
receptive environment for jazz to grow. Why New Orleans? It could be because
there were many slaves and cultures in this melting pot of a city. It could be
because it was a city that was fairly tolerant of many races and creeds.
Regardless, "America's Most Interesting City" gave birth to one of the most
interesting genres of music.

The New Orleans style of jazz was known for its collective improvisation. This
means that each instrument will play a unique, often improvised part. The
instruments that make up a Jazz band were different from any other type of music
group. The front-line consisted of the coronet, trombone, and clarinet.  The
back-line held the tuba, guitar, banjo, and eventually the drums. The melody was
given to the cornet.

After seeing its origins in New Orleans, jazz spread to Chicago. Chicago style
focused more on the soloists rather than the collective improve of New Orleans
style. One of the most famous cornet players was Louis Armstrong. A member of
King Oliver's band, Armstrong was a big influence in the shift from collective
improve to soloists. At this point in jazz, the piano and the saxophone were
added to the ensemble, achieving a much fuller sound.

Then along came the 1930's. Probably one of the most key decades of jazz. Many
of the most popular names in Jazz arose from this decade. Influential names like
Ella Fitzgerald, Duke Ellington, Benny Goodman, Fletcher Henderson, and Billie
Holiday served to pave the path for an era of jazz known as swing or big band.
An important thing to remember here, is that swing jazz is a type of dance
music. This allowed jazz to reach the ears of many brand new types of audiences.
Jazz was able to expand beyond the social restrictions of race. Many bandleaders
were prominent enough mingle with the upper echelons of white society.

During the 1940's, World War II hit. It became a little more difficult for big
bands to keep going. A shift occurred in jazz in favor of smaller groups. Out of
these smaller groups arose a genre of jazz known as Bebop. Bebop was faster
paced, much more improvised and is similar to modern jazz. Cool jazz was almost
the exact opposite of Bebop. Cool jazz was known for its slower tempo and
lighter tone. It often incorporated elements of classical music and was often
performed by white jazz musicians.

In the 1950's many different genres of jazz arose. For example, hard bop, modal
jazz and free jazz just to name a few. Hard Bop was a slow tempo jazz that often
used elements of gospel and blues. To many, it was seen as the black response to
the predominantly white, cool jazz. Two of the most important names in modal
jazz were John Coltrane, and my personal favorite, Miles Davis. Davis "modestly"
believed that he changed the course of music five times throughout his career.
It  was bigheaded free-thinking personalities like Davis that pioneered jazz to
take a direction towards freedom in structure and improvisation.

After jazz took on this artistic freedom, it eventually turned into Rhythm and
Blues, which many don't consider to be jazz at all. Overall, jazz remains one of
the most free-forming genres to ever exist. It has a uniquely American tale of
origin that no other genre of music can associate with. Heavily influenced by
African American culture and spirit, jazz has become a music that often
represents America in its golden age of musical revolution.

\end{document}
