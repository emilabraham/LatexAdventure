\documentclass{article} \begin{document}

\section{Question 1} Question 1: What are your personal and academic reasons for
wishing to participate in this Dialogue of Civilizations program?\\

% Answer to Question 1.
Personally, I have always wanted to visit intellectual high points in other
countries. For the most part, when I visit a new country, I travel with my
family and together, we visit the standard tourist locations. We walk the walk,
and we see the sights. For example, when I visited Dubai for a week, I saw many
attractions, which include the Burj Khalifa, Burj Al Arab, the man-made palm
islands, the famous indoor ski slope in the Mall of the Emirates. While all of
these sights were fascinating, and enjoyable to watch, not once did I experience
the brilliant intellectual legacy that the Arabs left us. I did not see a single
university, or museum, or place of learning. I marveled in the standard views
that every other person who walked my footsteps had seen. I wanted something
more. I wanted to see the tools that early Arab astronomers used to follow the
movements of the celestial bodies. I wanted to see the books that early
mathematicians wrote to explain the concept of zero. I yearned to appreciate the
contributions that these early intellectuals to allow us to come so far. This is
something that I hope this trip to Italy will help me to achieve. We will visit
places like the Museum of Computational Tools and Fondazione Scienza e Technica.
Both of which places that represent and portray significant human intellectual
advancements.

I also wish to experience the culture of this country without that layer of
tourist clouding my vision. When I typically visit another country, I typically
only stay for a week. That is not enough time to develop a schedule to
incorporate leisure time to truly \textit{live} in that country. To talk with
the locals, learn from them. That is what I believe it means to truly experience
a new culture.

\section{Question 2} Question 2: How will the program further your academic and
career goals?\\

% Answer to Question 2
I am a Computer Science major. Design algorithms, analyzing their complexity and
optimizing them is my thing. The morning class we will be taking every morning
in Italy will be catered to different revolutions in  scientific thought.
Professor Meleis described 2 periods in particular. The first was and era of
significant scientific progress. This was a period that scientists believe that
any problem can be solved with enough brain power and resources. The second was
an era in which scientists discovered that there were particular problems that
were extremely difficult to solve. There might even be a possibility that they
cannot be solved in an efficient manner. I believe understanding these types of
problems is a way to truly grasp the limitations of certain algorithms and
understanding the predicaments of computer science at its core. In Theory of
Computation, we discussed problems of NP completeness and undecidability
problems in relation to proving their theories. In this study abroad class, I
expect to see more concrete and real examples of the theories we discussed in
class. This will help solidify and connect the theories to the concrete. These
are problems that are tackled every day in back-end software design. Coming up
with the most efficient way to design and implement algorithms is a useful
skill, and knowing the upper bounds of those limitations can only help. 

\section{Question 3} Question 3: What is your previous travel and language
experience, if any?\\

%Answer to Question 3
Both my mother and father's side of the family are from Kerala. A state in the
southwestern portion of India. Kerala features a tropical, moist climate with a
dry season and a wet season. I travel to India about once every 5 years, in
order to visit relatives and attend weddings. We stay for about a month at a
time. Other than those trips to India, I have traveled to Dubai for a week for
vacation. I also spent 3 days in Cozumel Mexico as part of a week long cruise
vacation.

I am almost fluent in the local language of Kerala, Malayalam. I speak it well
enough that my relatives and family can understand me. I can also read and write
in Malayalam, which is a skill that many other American-born Malayalees my age
cannot boast of. I have my parents to thank for that. They encouraged me to
learn at a very young age. Other than Malayalam, I also know enough Spanish to
communicate and understand when I need to. Actually, in Cozumel, I took part in
a city-wide \textit{Amazing Race} style scavenger hunt. I had to use much of the
Spanish I learned to communicate with the locals and get a geographic bearing of
where our team was headed throughout the race. I have about 4 years of high
school level Spanish. I am also currently learning a little bit of Hindi through
a Northeastern Language learning program called NUCALLS. I have always been
fascinated with non-Latin based languages. 

\section{Question 4} Question 4: What courses have you taken which are directly
relevant to the program?\\

\end{document}
